\section{Pseudo-Seletores CSS}

Pseudo-seletores são utilizados para aplicar estilos a elementos com base em um estado ou característica específica, sem precisar modificar o HTML. Nesta seção, vamos explorar alguns pseudo-seletores comuns do CSS e como utilizá-los.

\subsection{:after}

O pseudo-seletor \var{:after} é utilizado para adicionar conteúdo após o conteúdo de um elemento. É comumente usado para adicionar elementos decorativos, como ícones e setas, a elementos de navegação. O código \ref{exemplo-after} mostra como adicionar uma seta após um link.

\begin{csscode}{Adicionando uma seta após um link.}{exemplo-after}
a:after {
    content: " >>";
}
\end{csscode}

Neste exemplo \ref{exemplo-after}, o conteúdo de \var{:after} é definido como uma seta usando a propriedade \var{content}. A seta é adicionada após o conteúdo do link.

\subsection{:before}

O pseudo-seletor \var{:before} é utilizado para adicionar conteúdo antes do conteúdo de um elemento. É comumente usado para adicionar elementos decorativos, como ícones e setas, a elementos de navegação.O código \ref{exemplo-before} mostra como adicionar uma seta antes de um link.

\begin{csscode}{Adicionando uma seta antes de um link.}{exemplo-before}
a:before {
    content: " >>";
}
\end{csscode}

Neste exemplo \ref{exemplo-before}, o conteúdo de \var{:before} é definido como uma seta usando a propriedade \var{content}. A seta é adicionada antes do conteúdo do link.

\subsection{:first-letter}

O pseudo-seletor \var{:first-letter} é utilizado para aplicar estilos apenas à primeira letra de um elemento. É comumente usado para adicionar destaque a letras iniciais em textos. O código \ref{exemplo-first-letter} mostra como aplicar um estilo de cor diferente à primeira letra de um parágrafo.

\begin{csscode}{Aplicando estilo à primeira letra de um parágrafo.}{exemplo-first-letter}
    p:first-letter {
    color: red;
}
\end{csscode}

Neste exemplo \ref{exemplo-first-letter}, o pseudo-seletor \var{:first-letter} é usado para aplicar a cor vermelha apenas à primeira letra do parágrafo.

\subsection{:first-line}

O pseudo-seletor \var{:first-line} é utilizado para aplicar estilos apenas à primeira linha de um elemento. É comumente usado para definir estilos para o início de um parágrafo ou título. O código \ref{exemplo-first-line} mostra como aplicar um estilo de cor diferente à primeira linha de um parágrafo.

\begin{csscode}{Aplicando estilo à primeira linha de um parágrafo.}{exemplo-first-line}
p:first-line {
    color: blue;
}
\end{csscode}

Neste exemplo \ref{exemplo-first-line}, o pseudo-seletor \var{:first-line} é usado para aplicar a cor azul apenas à primeira linha do parágrafo.

\subsection{li::marker}

O pseudo-seletor \var{li::marker} é utilizado para selecionar o marcador de um elemento de lista \var{<li>}. O marcador é a parte do elemento de lista que aparece antes do conteúdo do elemento. O código \ref{exemplo-li-marker} mostra como estilizar o marcador de uma lista com círculos vermelhos.

\begin{csscode}{Estilizando o marcador de uma lista.}{exemplo-li-marker}
li::marker {
    content: "";
    display: inline-block;
    width: 0.5em;
    height: 0.5em;
    margin-right: 0.5em;
    background-color: red;
    border-radius: 50%;
}
\end{csscode}

Neste exemplo \ref{exemplo-li-marker}, o pseudo-seletor \var{li::marker} é usado para selecionar o marcador de uma lista e aplicar um estilo de círculo vermelho a ele. A propriedade \var{content} é usada para remover o conteúdo padrão do marcador. Em seguida, a propriedade \var{display} é usada para exibir o marcador como um bloco em linha. As propriedades \var{width}, \var{height} e \var{border-radius} são usadas para definir o tamanho e a forma do círculo. A propriedade \var{margin-right} é usada para adicionar um espaço entre o marcador e o conteúdo do elemento de lista.

\subsection{Operadores \var{\textasciicircum=} e \var{\$=}}

Os operadores \var{\textasciicircum=} e \var{\$=} são utilizados para selecionar elementos cujos atributos começam com algo e que terminam com algo, respectivamente. O código \ref{exemplo-href} mostra como estilizar links que começam com \texttt{http} em negrito e links que terminam com \texttt{.pdf} em vermelho.

\begin{csscode}{Estilizando links específicos.}{exemplo-href}
a[href^="http"] {
    font-weight: bold;
}

a[href$=".pdf"] {
    color: red;
}
\end{csscode}

Neste exemplo \ref{exemplo-href}, os pseudo-seletores \var{[href\textasciicircum="http"]} e \var{[href\$=".pdf"]} são usados para selecionar links que começam com \texttt{http} e links que terminam com \texttt{.pdf}, respectivamente. Em seguida, o estilo é aplicado a cada um dos links selecionados.

\subsection{:not}

O pseudo-seletor \var{:not} é utilizado para selecionar elementos que não correspondem a um seletor específico. Ele é especialmente útil quando se deseja selecionar um conjunto de elementos, exceto um ou alguns específicos. O código \ref{exemplo-not} mostra como estilizar todos os links de um menu, exceto o primeiro.

\begin{csscode}{Estilizando todos os links de um menu, exceto o primeiro.}{exemplo-not}
    nav a:not(:first-of-type) {
    color: blue;
}
\end{csscode}

Neste exemplo \ref{exemplo-not}, o pseudo-seletor \var{:not} é usado para selecionar todos os links de um menu, exceto o primeiro. Em seguida, o estilo é aplicado a cada um dos links selecionados.

\subsection{:first-of-type e :not(:first-of-type)}

Os pseudo-seletores \var{:first-of-type} e \var{:not(:first-of-type)} são utilizados para selecionar o primeiro elemento de um tipo específico e todos os elementos desse tipo, exceto o primeiro, respectivamente. O código \ref{exemplo-first-of-type} mostra como estilizar o primeiro parágrafo de um artigo em negrito e todos os outros parágrafos em itálico.

\begin{csscode}{Estilizando o primeiro parágrafo de um artigo.}{exemplo-first-of-type}
article p:first-of-type {
    font-weight: bold;
}

article p:not(:first-of-type) {
    font-style: italic;
}
\end{csscode}

Neste exemplo \ref{exemplo-first-of-type}, o pseudo-seletor \var{:first-of-type} é usado para selecionar o primeiro parágrafo do artigo e aplicar um estilo de negrito a ele. Em seguida, o pseudo-seletor \var{:not(:first-of-type)} é usado para selecionar todos os outros parágrafos do artigo e aplicar um estilo de itálico a eles.

\subsection{input[type="tipo"] e input:focus}

Os pseudo-seletores \var{input[type="tipo"]} e \var{input:focus} são utilizados para selecionar elementos \var{<input>} com um tipo específico e o elemento \var{<input>} que está com o foco, respectivamente. O código \ref{exemplo-input} mostra como estilizar um campo de entrada de texto quando ele está com foco e um campo de entrada de senha.

\begin{csscode}{Estilizando campos de entrada de texto.}{exemplo-input}
input[type="text"]:focus {
    border-color: blue;
}

input[type="password"] {
    border: 1px solid black;
}
\end{csscode}

Neste exemplo \ref{exemplo-input}, o pseudo-seletor \var{input[type="text"]:focus} é usado para selecionar um campo de entrada de texto quando ele está com foco e aplicar um estilo de borda azul a ele. Em seguida, o pseudo-seletor \var{input[type="password"]} é usado para selecionar um campo de entrada de senha e aplicar um estilo de borda preta a ele.

\subsection{input:required, input:valid e input:invalid}

Os pseudo-seletores \var{input:required}, \var{input:valid} e \var{input:invalid} são utilizados para selecionar elementos \var{<input>} que são obrigatórios, válidos e inválidos, respectivamente. O código \ref{exemplo-input-validacao} mostra como estilizar um campo de entrada de texto que é obrigatório, válido e inválido.

\begin{csscode}{Estilizando campos de entrada de texto com validação.}{exemplo-input-validacao}
input:required {
    background-color: yellow;
}

input:valid {
    border-color: green;
}

input:invalid {
    border-color: red;
}
\end{csscode}

Neste exemplo \ref{exemplo-input-validacao}, o pseudo-seletor \var{input:required} é usado para selecionar um campo de entrada de texto que é obrigatório e aplicar um estilo de fundo amarelo a ele. Em seguida, o pseudo-seletor \var{input:valid} é usado para selecionar um campo de entrada de texto que é válido e aplicar um estilo de borda verde a ele. Finalmente, o pseudo-seletor \var{input:invalid} é usado para selecionar um campo de entrada de texto que é inválido e aplicar um estilo de borda vermelha a ele.

\subsection{input:in-range, input:out-of-range, input:disabled, input:checked, input:optional, input:readonly e input:read-write}

Os pseudo-seletores \var{input:in-range}, \var{input:out-of-range}, \var{input:disabled}, \var{input:checked}, \var{input:optional}, \var{input:readonly} e \var{input:read-write} são utilizados para selecionar elementos \var{<input>} que estão em um determinado estado, como habilitado, desabilitado, marcado, etc.

O código \ref{exemplo-input-estados} mostra como estilizar um campo de entrada de número que está habilitado, desabilitado, com um valor dentro e fora do intervalo, marcado e desmarcado.

\begin{csscode}{Estilizando campos de entrada de número com diferentes estados.}{exemplo-input-estados}
input:enabled {
    color: black;
}

input:disabled {
    color: gray;
}

input:in-range {
    border-color: green;
}

input:out-of-range {
    border-color: red;
}

input:checked {
    background-color: blue;
}

input:optional {
    border-color: yellow;
}

input:readonly {
    background-color: lightgray;
}

input:read-write {
    background-color: white;
}
\end{csscode}

Neste exemplo \ref{exemplo-input-estados}, o pseudo-seletor \var{input:enabled} é usado para selecionar um campo de entrada de número que está habilitado e aplicar um estilo de cor preta a ele. Em seguida, o pseudo-seletor \var{input:disabled} é usado para selecionar um campo de entrada de número que está desabilitado e aplicar um estilo de cor cinza a ele. Em seguida, o pseudo-seletor \var{input:in-range} é usado para selecionar um campo de entrada de número que está dentro do intervalo e aplicar um estilo de borda verde a ele. O pseudo-seletor \var{input:out-of-range} é usado para selecionar um campo de entrada de número que está fora do intervalo e aplicar um estilo de borda vermelha a ele. O pseudo-seletor \var{input:checked} é usado para selecionar um campo de entrada de número que está marcado e aplicar um estilo de fundo azul a ele. O pseudo-seletor \var{input:optional} é usado para selecionar um campo de entrada de número que é opcional e aplicar um estilo de borda amarela a ele. O pseudo-seletor \var{input:readonly} é usado para selecionar um campo de entrada de número que é somente leitura e aplicar um estilo de fundo cinza claro a ele. Por fim, o pseudo-seletor \var{input:read-write} é usado para selecionar um campo de entrada de número que é de leitura e escrita e aplicar um estilo de fundo branco a ele.

Esta seção apresentou alguns dos pseudo-seletores mais comuns em CSS, com exemplos de código e situações de uso para cada um deles. É importante lembrar de utilizar os comandos de latex \var{NomeDoTermo} para os termos em inglês e palavras-reservadas da CSS, bem como utilizar as referências com os códigos dos exemplos para facilitar a compreensão do leitor.